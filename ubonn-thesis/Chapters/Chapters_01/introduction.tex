% !TEX root = mythesis.tex

%==============================================================================
\chapter{Introduction}
\label{sec:intro}
%==============================================================================
The top quark is the most massive particle in the standard model (SM) of particle physics, which arises predominantly from the production of top quark-antiquark ($t\Bar{t}$) pairs through the strong interaction. However top quarks may also be produced singly from electroweak processes: t-channel, s-channel and associated tW production. The D0 and CDF collaborations \cite{PhysRevLett.103.092001,PhysRevLett.103.092002} as well as the ATLAS \cite{ATLAS:2012byx} and CMS \cite{PhysRevLeCMS} collaborations have measured the cross-sections for single top quark production. The high integrated luminosity and the center-of-mass energy at the LHC allows the study of processes with very small cross-sections that were not accessible at lower energies such as the production of a single top quark in association with a Z boson. In this production mechanism the top quark is produced via the t-channel and the Z boson is either radiated off from one of the participating quarks or produced via W boson fusion leading to a signature with a single top quark, a Z boson and an additional quark. The process is sensitive to top quark couplings to the Z boson and also to the triple gauge-boson coupling (WWZ). The measurement of tZq production are also sensitive to processes beyond the SM.

In this thesis, the tZq total cross-section measurement are performed in the trilepton final states, where both the W boson from the top quark and the Z boson decay into either electrons or muons, resulting in four possible leptonic combinations in the final state: $eee$, $ee\mu$, $e\mu\mu$, $\mu\mu\mu$ and there is also a contributions from leptonic $\tau$ decays. The thesis is structured as follows: chapter \ref{chap:theory} presents an overview of the SM of particle physics. It also presents the detail on top quark physics including the rare single top production processes. Chapter \ref{chap:lhcandatlas} explains about the Large Hadron Collider, ATLAS experiment and describes how fundamental physics objects are reconstructed from the detector information. The data and the Monte Carlo simulated samples that are used for modelling the signal and background processes are described in chapter \ref{chap:data_MC}. Chapter \ref{chap:event_selection} begins the description of the final state of the tZq process. Several signal regions and control regions are also defined to constrain the backgrounds, each containing different contributions from signal and background processes. It also  presents analysis of the lepton working point and optimization of selection cuts. In chapter \ref{chap:multivariate_analysis}, the measurements are performed based on multivariate analysis where artificial neural networks (NN) training is used to enhance the signal-to-background separation. The sources of systematic uncertainties that need to be taken into account for the final cross-section measurement are also presented. A binned likelihood fit is performed to extract the signal strength, which is explained in the last section. A discussion of the fit results obtained using both an asimov dataset and the real data is included in chapter \ref{chap:results}. The thesis is concluded in chapter \ref{chap:conclusion}.
